\documentclass[
    ,paper=a4
    ,fontsize=12pt
    ,bibliography=totoc
    ,sfdefaults=false
    ,headings=small
    ,draft=false
]{scrartcl}

% This allows to type UTF-8 characters like ä,ö,ü,ß
\usepackage[ngerman]{babel}
\usepackage[autostyle,german=guillemets]{csquotes} % Gives german guillemet-quotation marks like in »Quoted text.«

\usepackage[T1]{fontenc}        % Tries to use Postscript Type 1 Fonts for better rendering
\usepackage[
    ,semibold
    ,tabular
    ,default
    ,osf
]{sourcesanspro}                % Provides the Source Sans Pro font family
\usepackage[final]{microtype}
\usepackage{setspace}
\usepackage[margin=3.3cm, left=2.5cm, right=2.5cm, includeheadfoot]{geometry} % für Seitenränder
% \usepackage[margin=2.5cm,includeheadfoot]{geometry}

\usepackage{tabularray}
\usepackage{longtable}
\usepackage{ragged2e}
\usepackage{listings}

\usepackage[
backend=biber,
style=numeric,
sorting=nyt
]{biblatex}
\addbibresource{projekt.bib}

\usepackage{enumitem}
\setlist{noitemsep}

%\usepackage{grffile}           % Allows you to include images (like graphicx). Usage: \includegraphics{path/to/file}

% Additional packages
\usepackage{url}                % Lets you typeset urls. Usage: \url{https://...}
%\usepackage{breakurl}          % Enables linebreaks for urls
\usepackage{xspace}             % Use \xpsace in macros to automatically insert space based on context. Usage: \newcommand{\es}{ESPResSo\xspace}
\usepackage[svgnames]{xcolor}   % Obviously colors. Usage: \color{red} Red text
\usepackage{booktabs}           % Nice rules for tables. Usage \begin{tabular}\toprule ... \midrule ... \bottomrule\end{tabular}
\usepackage{graphicx}           % A nice way to include images (see above: grffile)
% \usepackage{blindtext}          % Provides commands to generate text so you can see how your document will look

\usepackage[hang]{footmisc}     % Gives you "hanging" footnotes. Usage \footnote{text}
\deffootnote{1.5em}{0em}{\thefootnotemark\quad}

\usepackage[
    colorlinks=true,
    linkcolor=Teal,
    urlcolor=Teal,
    breaklinks=true,
]{hyperref}                     % Provides clickable links in the PDF-document for \ref

\usepackage{bookmark}

\begin{document}
% \sloppy

% \setstretch{1.15}
% !TEX root = projektdokumentation.tex
\begin{titlepage}
    \begin{center}
        \includegraphics[width=.5\textwidth]{images/ihk-logo.jpg}

        \vspace{1cm}
        \Large
        Dokumentation zur betrieblichen Projektarbeit\\[.25em]
        — Mittelstufenprojekt —\\
        \vspace{1cm}
        \huge
        \textbf{BotHQ – ein Discord-Bot-Framework}
            
        \vspace{0.5cm}
        \LARGE
        Untertitel
            
        \vfill   
        \begin{table}[h]
        \large
\begin{tblr}{ll}
Prüfungsausschuss:  & \begin{tabular}[t]{@{}l@{}}
                    Bernd Heidemann\\
                    Katrin Deeken\end{tabular}
                    \\[.5em]
Abgabedatum:        & 29. Mai 2024\\[.5em]
Prüfungsbewerber:   & \begin{tabular}[t]{@{}l@{}}
                    Philipp Batelka\\
                    Jan Mahnken\\
                    Daniel Quellenberg\\
                    Fabian Reichwald\\
                    Justus Sieweke\\
                    Christopher Spencer\end{tabular} \\[.5em]
Ausbildungsbetrieb: & \begin{tabular}[t]{@{}l@{}}
                    Europaschule Schulzentrum SII Utbremen\\
                    Meta-Sattler-Str. 33\\ 
                    28217 Bremen\end{tabular}
\end{tblr}
\end{table}
            
    \end{center}
\end{titlepage}
\phantomsection
\pagenumbering{Roman}
\pdfbookmark[1]{Inhaltsverzeichnis}{inhalt}
\tableofcontents

\cleardoublepage

\phantomsection
\listoffigures
\cleardoublepage

\phantomsection
\listoftables
\cleardoublepage

\phantomsection
\lstlistoflistings
\cleardoublepage

\clearpage
\pagenumbering{arabic}
\section{Einleitung}
\blindtext
\subsection{Projektumfeld}
\subsection{Projektziel}
\subsection{Projektbegründung}
\subsection{Projektschnittstellen}
\subsection{Projektabgrenzung}
\section{Projektplanung}
\blindtext
\subsection{Projektphasen}
\subsection{Abweichungen vom Projektantrag}
\subsection{Ressourcenplanung}
\subsection{Entwicklungsprozess}
\section{Analysephase}
\blindtext
\subsection{Ist-Analyse}
\subsection{Wirtschaftlichkeitsanalyse}
\subsection{„Make or Buy“-Entscheidung}
\subsection{Projektkosten}
\subsection{Amortisationsdauer}
\subsection{Nutzwertanalyse}
\subsection{Anwendungsfälle}
\subsection{Qualitätsanforderungen}
\subsection{Lastenheft/Fachkonzept}
\input{contents/entwurfsphase}
% !TEX root = ../projektdokumentation.tex

\section{Implementierungsphase}
\blindtext
\subsection{Implementierung der Datenstrukturen}
\subsection{Implementierung der Benutzeroberfläche}
\subsection{Implementierung der Geschäftslogik}
% !TEX root = ../projektdokumentation.tex

\section{Abnahmephase}
\blindtext
% !TEX root = ../projektdokumentation.tex

\section{Einführungsphase}
\blindtext
% !TEX root = ../projektdokumentation.tex

\section{Dokumentation}
\blindtext
\section{Fazit}
\blindtext
\subsection{Soll-/Ist-Vergleich}
\subsection{Lessons Learned}
\subsection{Ausblick}

% \nocite{*}
% \printbibliography

\end{document}