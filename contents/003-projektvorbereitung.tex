% !TEX root = ../projektdokumentation.tex

\section{Projektvorbereitung}\label{projektvorbereitung}

\subsection{Ist-Analyse}\label{ist-analyse}
Derzeit verfügt die HiTec GmbH nicht über ein Discord-Bot-Framework oder anderweitige Dienstleistungen in bezug auf Discord. Dies hat zur Folge, dass die HiTec Gmbh keine Kunden Betreuen kann, die Discord zur Kommunikation nutzen, weder Intern noch mit Kunden.

Desweiterem wäre die HiTec Gmbh bei eigener Nutzung von Discord zu Kommunikationszwecken von Dienstleistungen von Drittanbietern abhängig, um die Anwendungsmöglichkeiten von Discord für den eigenen Anwendungsfall anzupassen und zu optimieren.

\subsection{Soll-Analyse}\label{soll-analyse}
Soll-Zustands ist es, das ein Cloud Service der HiTec Gmbh für Discord-Bots existieren soll, das auf der \gls{JDA} \autocite{jda-github,jda-wiki} basiert. Der Service soll  Nutzern über eine Weboberfläche ermöglichen, Plugins zu Discord-Servern hinzuzufügen und zu verwalten. Desweiteren soll ein Framework entwickelt werden, welches ein einfaches hinzufügen von neuen Plugins von Seiten der HiTec Gmbh ermöglicht. Dies erlaubt es der HiTec Gmbh eine modulare Dienstleistung für Kunden, die Discord zu Kommunikationszwecken nutzen, anzubieten und diese auf Wunsch der Kunden einfach anzupassen. 

\subsection{Projektziel detailliert}\label{projektziel-detailliert}
Der Service soll dem Kunden als Webservice zur verfügung gestellt werden, damit er die Verwaltung seiner Server und Plugins selbstständig durführen kann. Die Platform soll responsive gestaltetwerden mit dem Fokus auf einem benutzerfreundlichen Design und fehlerfreier Funktionalität. 

Beim öffnen der Webseite soll der Nutzer dazu aufgefordert werden, sich zu authentifizieren. Der Service zur authentifizierung wird hierbei von der Discord-API bereit gestellt.

Auf der Website soll der Nutzer eine Liste Seiner Discord-Server und der verfügbaren Plugins angezeigt bekommen. Die Liste der Server wird von der Discord-API bereitgestellt, die Liste der Plugins wird wiederum von der eigenen Backend-API des Services. Nutzern soll es möglich sein, Plugins und die Einstellungen dieser für jeden Server einzeln und in Echtzeit verwalten zu können. Eine Datenbank im Backend des Services Speichert die so vorgenommenen Einstellungen ab. Das Backend soll die vom Nutzer vorgenommenen Einstellungen automatisiert vornehmen. 

Desweiteren soll ein Framework zum erstellen von Plugins für den Inhaber des Services existieren. Einem Entwickler soll es ermöglicht werden, Code schreiben zu können, der mit einem Dicord-Server interagieren kann. Zusätzlich soll der Entwickler eine Konfigurationsdatei bereitstellen können. Diese soll Parameter enthalten, die vom Endnutzer manipuliert können sollen. Hierzu wird die Konfigurationsdatei beim Aufruf der Plugin-Einstellungen durch den nutzer an das Frontend übermittel, was dann daraus dynamisch ein Interface für den Nutzer generiert.

\subsection{Anforderungen und Technologien}\label{anforderungen-und-technologien}
Zur Durführung des Projektes ist die Nutzung der folgenden Tools und Technologien vorgesehen:

\begin{itemize}
 \item \textbf{Java Spring Framework} ist ein Open-Source Framework zur Entwicklung von eigenständigen Anwendungen, die auf der Java Virtual Machine (JVM) laufen. Spring bietet eine Abhängigkeitsinjektionsfunktion, bei der Objekte ihre eigenen Abhängikkeiten definieren können, und diese dann später injiziert werden können. Dadurch können modulare Anwendungen aus lose verbundenen Komponenten erstellt werden, was sich sehr gut für Netzanwendungen und Mikroservices eignet. Zusätzlich beinhaltet das Framework auch Funktionen und Unterstützungen für klassiche aufgaben wie zum Beispiel Ausnahmebehandlung, Typumsetzung, Internationalisierung und Weitere,
 \item \textbf{Java Spring Boot} ist ein Tool, dass die Entwicklung von Anwendungen mit dem Java Spring Framework durch drei Kernfunktionen vereinfacht:
    \begin{itemize}
        \item Autokonfiguration
        \item Ein „Opinionated"-Konfigurationsansatz
        \item Die Fähigkeit, eigenständige Anwendungen zu erstellen
    \end{itemize}
 \item \textbf{Eine Rest API} ist ein Interface, das den Beschränkungen von REST unterliegt. REST steht für „Representational State Transfer“ und wurde vom US-amerikanischen Informatiker Roy Fielding entwickelt. Es handelt sich dabei um eine Sammlung von Architekturbeschränkungen, mit einem Schwerpunkt auf Maschine-zu-Maschine Kommunikation. Bei einer Anforderung über eine Rest-API werden Ressourcen in Form eines HTTP-Formates an einen Client oder Endpunkt geschickt.
 Restful-APIs zeichnen sich durch folgende Merkmale aus:
 \begin{itemize}
     \item Client/Server-Architektur, die Anforderungen per HTTP verwaltet
     \item Zustandslose Client/Server-Kommunikation
     \item Cachingfähige Daten
     \item eine Einheitliche Schnitttstelle
 \end{itemize}
 \item \textbf{Angular} ist ein auf Typescript basierendes Front-End Framework zur entwicklung von Webapplicationen. Es wird als Open-Source-Plattform von einer Community aus Einzelpersonen und Firmen, angeführt durch Google, entwickelt.
 Angular verwendet eine Komponentenarchitektur, um Webanwendungen zu erstellen, bei der jeder Teil der Anwendung in unabhängige, wiederverwendbare Komponenten aufgeteilt wird. Zusätzlich liefert Angular viele Tools zur Webentwicklung, sowie Routing- und Navigationssysteme, eineleistungsfähige Templatingsprache und weitere. Vorteile von Angular sind unter anderem:
 \begin{itemize}
     \item Reduzierter Code
     \item Einfache Refaktorierung
     \item gute Testbarkeit
     \item Wiederverwendbare Code-Komponenten
     \item Große Community und ständige Weiterentwicklung
 \end{itemize}
 \item \textbf{Postgresql} , allgemein „Post-GRES“ ausgesprochen, ist eine Open-Source-Datenbank, die für ihre Zuverlässigkeit, Flexibilität und Unterstützung offener technischer Standards bekannt ist. Im Gegensatz zu anderen RDMBS (Relational Database Management Systems) unterstützt PostgreSQL sowohl nicht-relationale als auch relationale Datentypen. Dies macht sie zu einer der konformsten, stabilsten und ausgereiftesten relationalen Datenbanken, die aktuell verfügbar sind.Es wurde ursprünglich 1986 als Nachfolger von INGRES (einem in den frühen 1970er Jahren begonnenen Open-Source-SQL-Projekt für relationale Datenbanken) entwickelt und war die Idee von Michael Stonebraker, einem Informatikprofessor in Berkeley.
 \item \textbf{Discord} ist ein Service für die Online-Kommunikation, Instant Massaging, Sprach- und Vidoekonferenzen. Es ist sowohl als Webanwendung als auch als Client Software auf allen gängigen Betriebssystemen genutzt werden. Ursprünglich war Discord vornehmend für die Videospiel-Szene gedacht, um während des Spielens Informationen mit Mitspielern austauschen zu können. Inzwischen wird Discord jedoch von einer großen Anzahl an Online-Communitys genutzt. Das Konzept von Discord stammt von Jason Citron, der OpenFeint, eine Social-Gaming-Plattform für mobile Spiele, gegründet hatte, und von Stanislav Vishnevsky, dem Gründer von Guildwork, einer weiteren Social-Gaming-Plattform.
 \item \textbf{Die Java Discord API (JDA)} ist eine umfassende Programmbibliothek, die Entwicklern ermöglicht, Discord-Bots in der Programmiersprache Java zu erstellen und zu verwalten. JDA bietet eine hohe Abstraktionsebene, um die Kommunikation zwischen einem Java-Programm und der Discord-API zu erleichtern. Mit JDA können Entwickler auf verschiedene Funktionen und Ereignisse in Discord zugreifen, wie das Empfangen und Senden von Nachrichten, das Verwalten von Servern, Kanälen und Benutzern sowie das Reagieren auf verschiedene Interaktionen innerhalb von Discord.
        
        Zu den Hauptfunktionen der JDA gehören:
        \begin{itemize}
            \item \textbf{Nachrichtenverwaltung}: Senden, Bearbeiten und Löschen von Nachrichten in Textkanälen.
            \item \textbf{Benutzerverwaltung}: Abrufen von Benutzerinformationen, Verwalten von Rollen und Berechtigungen.
            \item \textbf{Ereignisbehandlung}: Reagieren auf Ereignisse wie Nachrichtenempfang, Benutzerbeitritt und -austritt, Reaktionen und vieles mehr.
            \item \textbf{Sprachunterstützung}: Unterstützung für die Audioübertragung in Sprachkanälen.
        \end{itemize}
        
        JDA ist besonders für seine einfache Handhabung und umfangreiche Dokumentation bekannt, die es sowohl Anfängern als auch erfahrenen Entwicklern ermöglicht, leistungsstarke und funktionsreiche Discord-Bots zu erstellen. Die Bibliothek wird aktiv gepflegt und weiterentwickelt, um stets mit den neuesten Änderungen und Funktionen der Discord-API kompatibel zu bleiben.
\end{itemize}