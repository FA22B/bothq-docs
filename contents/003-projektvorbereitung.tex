% !TEX root = ../projektdokumentation.tex

\section{Projektvorbereitung}\label{projektvorbereitung}

\subsection{Ist-Analyse}\label{ist-analyse}

\begin{itemize}
  \item
        Beschreibung der gegenwärtigen Situation:
  \item
        Welche Probleme oder Herausforderungen bestehen derzeit?
  \item
        Was ist der aktuelle Stand der Technik bzw. des Systems?
\end{itemize}

\subsection{Soll-Analyse}\label{soll-analyse}

\begin{itemize}
  \item
        Detaillierte Beschreibung der Projektziele:
  \item
        Was soll am Ende des Projektes erreicht sein?
  \item
        Welche Verbesserungen und Neuerungen sind geplant?
\end{itemize}

\subsection{Projektziel detailliert}\label{projektziel-detailliert}

\begin{itemize}
  \item
        Ausführliche Darstellung der Projektziele:
  \item
        Beschreibung der funktionalen und nicht-funktionalen Anforderungen.
\end{itemize}

\subsection{Anforderungen und Technologien}\label{anforderungen-und-technologien}

\begin{itemize}
  \item
        Auflistung und Beschreibung der Anforderungen:
  \item
        Technologische Anforderungen (z.B. REST-API, Datenbanken)
  \item
        Zielplattformen und benötigte Technologien (z.B. Cloud-Services)
\end{itemize}

Die Java Discord API (JDA) ist eine umfassende Programmbibliothek, die Entwicklern ermöglicht, Discord-Bots in der Programmiersprache Java zu erstellen und zu verwalten. JDA bietet eine hohe Abstraktionsebene, um die Kommunikation zwischen einem Java-Programm und der Discord-API zu erleichtern. Mit JDA können Entwickler auf verschiedene Funktionen und Ereignisse in Discord zugreifen, wie das Empfangen und Senden von Nachrichten, das Verwalten von Servern, Kanälen und Benutzern sowie das Reagieren auf verschiedene Interaktionen innerhalb von Discord.
        
        Zu den Hauptfunktionen der JDA gehören:
        \begin{itemize}
            \item \textbf{Nachrichtenverwaltung}: Senden, Bearbeiten und Löschen von Nachrichten in Textkanälen.
            \item \textbf{Benutzerverwaltung}: Abrufen von Benutzerinformationen, Verwalten von Rollen und Berechtigungen.
            \item \textbf{Ereignisbehandlung}: Reagieren auf Ereignisse wie Nachrichtenempfang, Benutzerbeitritt und -austritt, Reaktionen und vieles mehr.
            \item \textbf{Sprachunterstützung}: Unterstützung für die Audioübertragung in Sprachkanälen.
        \end{itemize}
        
        JDA ist besonders für seine einfache Handhabung und umfangreiche Dokumentation bekannt, die es sowohl Anfängern als auch erfahrenen Entwicklern ermöglicht, leistungsstarke und funktionsreiche Discord-Bots zu erstellen. Die Bibliothek wird aktiv gepflegt und weiterentwickelt, um stets mit den neuesten Änderungen und Funktionen der Discord-API kompatibel zu bleiben.