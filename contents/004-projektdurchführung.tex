% !TEX root = ../projektdokumentation.tex

\section{Projektdurchführung}\label{projektdurchfuxfchrung}

% \section{Entwurfsphase}
% \subsection{Zielplattform}
% \subsection{Architekturdesign}
% \subsection{Entwurf der Benutzeroberfläche}
% \subsection{Datenmodell}
% \subsection{Geschäftslogik}
% \subsection{Maßnahmen zur Qualitätssicherung}
% \subsection{Pflichtenheft/Datenverarbeitungskonzept}

\subsection{Vorgehensmodell}\label{vorgehensmodell}

\begin{itemize}
  \item
        Wahl des Vorgehensmodells (z.B. Scrum, Kanban):
  \item
        Begründung für die Wahl des Vorgehensmodells
  \item
        Auswirkungen des Vorgehensmodells auf die Projektdurchführung
\end{itemize}

\subsection{Umsetzung}\label{umsetzung}

\subsubsection{Klassenmodell}\label{klassenmodell}

\begin{itemize}
  \item
        Darstellung des Klassenmodells:
  \item
        Diagramme und Beschreibungen der Klassen und deren Beziehungen
\end{itemize}

\subsubsection{Datenhaltung}\label{datenhaltung}

\begin{itemize}
  \item
        Wahl der Datenhaltung:
  \item
        Beschreibung der Datenbankstruktur und des Datenbankmodells
\end{itemize}

\subsubsection{Design Patterns}\label{design-patterns}

\begin{itemize}
  \item
        Einsatz von Design Patterns:
  \item
        Welche Design Patterns werden verwendet und warum?
\end{itemize}

\subsection{Qualitätssicherung}\label{qualituxe4tssicherung}

\subsubsection{Teststrategien}\label{teststrategien}

\begin{itemize}
  \item
        Beschreibung der Teststrategien:
  \item
        Welche Tests werden durchgeführt (Unit Tests, Integrationstests, Systemtests)?
\end{itemize}

\subsubsection{Testfälle}\label{testfuxe4lle}

\begin{itemize}
  \item
        Konkrete Testfälle:

        \begin{itemize}

          \item
                Beschreibung der durchgeführten Testfälle und deren Ergebnisse
        \end{itemize}
\end{itemize}