% !TEX root = ../projektdokumentation.tex

\section{Wirtschaftliche Betrachtung}\label{wirtschaftliche-betrachtung}

\subsection{Marktuntersuchung}\label{marktuntersuchung}

\subsubsection{Zielgruppe}\label{zielgruppe}

Die Hauptzielgruppen für BotHQ lassen sich in mehrere Kategorien einteilen. Zunächst sind da die Gamer und Gaming-Communities. Discord ist bei Gamern sehr beliebt, und viele Gaming-Communities verwenden Bots, um Server zu verwalten, Spielinformationen bereitzustellen, Events und Turniere zu organisieren und vieles mehr.

Des Weiteren zählen technikbegeisterte Personen und Entwickler zur Zielgruppe. Diese Personen interessieren sich für die Programmierung und Entwicklung von Bots, verfügen jedoch möglicherweise nicht über die technischen Kenntnisse, um einen Discord-Bot von Grund auf zu erstellen und zu konfigurieren.

Auch Unternehmen und Organisationen, die Discord für die Teamkommunikation nutzen, sind eine wichtige Zielgruppe. Diese Unternehmen könnten Bots zur Automatisierung von Aufgaben, für den Kundenservice und andere organisatorische Zwecke einsetzen. Schließlich gehören auch Content Creator und Streamer zu den Zielgruppen. Streamer auf Plattformen wie Twitch und YouTube nutzen häufig Discord-Server, um mit ihrer Community zu interagieren, und könnten Bots zur Moderation und zur Steigerung des Engagements einsetzen.

Die Erwartungen der Zielgruppe an BotHQ umfassen eine einfache Installation und Einrichtung, Benutzerfreundlichkeit, vielfältige Plugins und Erweiterungen, Zuverlässigkeit und Stabilität, Sicherheit, guten Kundensupport und Dokumentation, Flexibilität und Anpassbarkeit sowie die Integration mit anderen Diensten.

\subsubsection{Marktvolumen und Marktpotential}\label{marktvolumen-und-marktpotential}

Um das Marktvolumen für BotHQ zu bestimmen, betrachten wir zunächst die Nutzerbasis von Discord, die weltweit über 250 Millionen registrierte Nutzer und über 100 Millionen aktive Nutzer pro Tag umfasst. Besonders beliebt ist die Plattform bei Gamern, Technik-Enthusiasten und Community-Organisatoren. Es wird geschätzt, dass etwa 30\% der Discord-Server mindestens einen Bot verwenden. Geht man davon aus, dass 30\% der aktiven Discord-Nutzer Bots verwenden, ergibt sich ein potenzielles Marktvolumen von ca. 30 Millionen aktiven Nutzern weltweit, die an BotHQ interessiert sein könnten.

Der Markt für Discord-Bots weist zudem ein erhebliches Wachstumspotenzial auf. Mit der zunehmenden Verbreitung von Discord in verschiedenen Communities und Branchen steigt auch die Nachfrage nach Bots, die bestimmte Aufgaben automatisieren und Server effizienter verwalten können. In vielen Bereichen wächst der Bedarf an Automatisierungslösungen, und Discord-Bots können eine Vielzahl von Aufgaben übernehmen, von der Moderation über die Bereitstellung von Informationen bis hin zur Interaktion mit der Community.

Zusätzlich eröffnen neue Entwicklungen in der API-Integration sowie die zunehmende Marktexpansion von Discord neue Möglichkeiten für BotHQ. Während Discord ursprünglich vor allem von Gamern genutzt wurde, hat sich die Plattform inzwischen auf andere Bereiche wie Bildung, Unternehmenskommunikation und soziale Netzwerke ausgeweitet. Diese neuen Marktsegmente bieten weitere Wachstumsmöglichkeiten für BotHQ. Innovative Geschäftsmöglichkeiten durch Premium-Dienste, personalisierte Lösungen und Support-Dienste können zusätzliche Einnahmequellen erschließen und die Nutzerbindung erhöhen.

\subsubsection{Konkurrenzanalyse}\label{konkurrenzanalyse}

Im Bereich der Discord-Bot-Management-Plattformen gibt es mehrere Konkurrenzprodukte, die ähnliche Dienste anbieten. Einige der wichtigsten Konkurrenten sind Dyno Bot, Mee6, Carl-bot, GAwesome Bot, Nightbot und BotGhost. Diese Bots bieten umfangreiche Verwaltungsfunktionen und zeichnen sich durch verschiedene Stärken und Schwächen aus.

BotHQ bietet jedoch mehrere einzigartige Funktionen und Vorteile, die es von der Konkurrenz abheben. Im Gegensatz zu vielen Konkurrenzprodukten, die eine manuelle Konfiguration erfordern, bietet BotHQ eine vollständig automatisierte Installation und Konfiguration von Bots und Plugins, was die Benutzerfreundlichkeit deutlich erhöht. Darüber hinaus ermöglicht BotHQ eine einfache Aktivierung und Konfiguration von Plugins, ohne den Bot neu starten oder Dateien manuell bearbeiten zu müssen. Eine intuitive und leicht zu navigierende Benutzeroberfläche stellt sicher, dass auch Nutzer ohne technische Vorkenntnisse den vollen Funktionsumfang des Bots nutzen können.

BotHQ legt großen Wert auf Sicherheit und Datenschutz und bietet regelmäßige Updates und Sicherheitspatches an, um sicherzustellen, dass die Bots und die Daten der Nutzer geschützt sind. Der umfangreiche Kundensupport und die aktive Community von BotHQ bieten den Nutzern bei Problemen und Fragen Unterstützung. Zusätzlich unterstützt BotHQ eine Vielzahl von Plugins und ermöglicht es Entwicklern, eigene Plugins zu erstellen und zu integrieren, um die Funktionalität des Bots kontinuierlich zu erweitern.

\subsection{Marketingmix}\label{marketingmix}

Die Produktpolitik von BotHQ umfasst verschiedene Produktvarianten, um den unterschiedlichen Bedürfnissen der Zielgruppen gerecht zu werden. Dazu gehören eine kostenlose Basisversion und kostenpflichtige Premiumversionen mit erweiterten Funktionen.

Die Preispolitik von BotHQ verfolgt eine Freemium-Preisstrategie. Diese Strategie ermöglicht es, eine breite Nutzerbasis zu erreichen und gleichzeitig durch Premium-Dienste Einnahmen zu generieren. 

In der Kommunikationspolitik setzt BotHQ auf verschiedene Kanäle, darunter Social Media, Influencer-Marketing, Partnerschaften mit Discord-Communities und gezielte Online-Werbung. Ein starkes Augenmerk wird zudem auf Content-Marketing gelegt, um die Sichtbarkeit und das Engagement zu erhöhen.

Die Distributionspolitik von BotHQ konzentriert sich hauptsächlich auf den Online-Vertrieb über die eigene Website sowie über Plattformen wie GitHub und Discord selbst. Diese Kanäle ermöglichen eine breite Verfügbarkeit und einfache Zugänglichkeit für die Nutzer.

\subsection{Kostenplanung}\label{kostenplanung}

Im Folgenden werden die potenziellen Kosten, die im Laufe des Projekts entstehen können, im Detail erläutert. Dies betrifft die Berechnung der Personalkosten, die jeweiligen Stundensätze, die Kosten für Sachmittel und die Gesamtkosten für die Durchführung des Projekts.

\subsubsection{Personalkosten}\label{personalkosten}

Das Projekt startete am 14. Februar 2024 und endet am 30. Mai 2024. Die Arbeitszeit ist für jeden Mittwoch von 8:10 bis 13:20 Uhr mit zwei Pausen je 20 Minuten geplant. Dies ergibt eine Arbeitszeit pro Mittwoch von 4 Stunden und 30 Minuten (270 Minuten) abzüglich 40 Minuten Pause, also 3 Stunden und 50 Minuten (230 Minuten oder 3.83 Stunden). Der Zeitraum umfasst insgesamt 16 Wochen.

\begin{table}[ht]
  \centering
  \begin{tabular}{lr}
    \toprule
    \textbf{Person} & \textbf{Stundensatz (EUR)} \\
    \midrule
    Fabian Reichwald & 50 \\
    Phillipp Batelka & 45 \\
    Justus Sieweke & 55 \\
    Daniel Quellenberg & 40 \\
    Jan Mahnken & 50 \\
    Christopher Spencer & 60 \\
    \bottomrule
  \end{tabular}
  \caption{Stundensätze der Projektmitglieder}
\end{table}

Die Berechnung der Personalkosten erfolgt wie folgt:

\[
  \begin{aligned}
    \text{Personalkosten pro Woche} &= (\text{Fabian Reichwald} + \text{Phillipp Batelka} + \text{Justus Sieweke} + \\
    &\quad \text{Daniel Quellenberg} + \text{Jan Mahnken} + \text{Christopher Spencer}) \\
    &\quad \times \text{Arbeitszeit pro Woche} \\
    &= (50 + 45 + 55 + 40 + 50 + 60) \times 3.83 \\
    &= 300 \times 3.83 \\
    &= 1149 \text{ EUR/Woche}
  \end{aligned}
\]

\begin{table}[ht]
  \centering
  \begin{tabular}{lr}
    \toprule
    \textbf{Gesamtkosten pro Woche} & 1149 EUR \\
    \midrule
    \textbf{Anzahl der Wochen} & 16 \\
    \midrule
    \textbf{Gesamtkosten} & 18384 EUR \\
    \bottomrule
  \end{tabular}
  \caption{Gesamtkosten pro Woche}
\end{table}

Die gesamten Personalkosten für 6 Personen über den Zeitraum von 3 Sprints (16 Wochen) betragen somit \textbf{18.384 EUR}.

\subsubsection{Sachmittelkosten}\label{sachmittelkosten}

Für die Durchführung des Projekts fallen auch Sachmittelkosten an. Diese umfassen Hardware-, Soft\-ware- und Zusatzkosten.

\begin{table}[ht]
  \centering
  \begin{tabular}{lrr}
    \toprule
    \textbf{Hardware} & \textbf{Anzahl} & \textbf{Kosten (EUR)} \\
    \midrule
    Laptop (Windows) & 2 & 2000 \\
    Laptop (Linux) & 2 & 1800 \\
    Laptop (Mac) & 2 & 3000 \\
    \midrule
    \textbf{Gesamtkosten Hardware} & & 6800 \\
    \bottomrule
  \end{tabular}
  \caption{Hardware-Kosten}
\end{table}

Um den Entwicklern flexible und leistungsfähige Arbeitsmittel zur Verfügung zu stellen, werden 6 Laptops benötigt. Verschiedene Betriebssysteme (Windows, Linux, Mac) sind erforderlich, um sicherzustellen, dass die entwickelten Anwendungen auf verschiedenen Plattformen getestet und optimiert werden können.

\begin{table}[ht]
  \centering
  \begin{tabular}{lrr}
    \toprule
    \textbf{Software} & \textbf{Anzahl} & \textbf{Kosten (EUR)} \\
    \midrule
    JetBrains Lizenz (Full Package) & 6 & 1500 \\
    Webhosting & 1 & 200 \\
    Domain & 1 & 10 \\
    SSL-Zertifikat & 1 & 50 \\
    \midrule
    \textbf{Gesamtkosten Software} & & 1760 \\
    \bottomrule
  \end{tabular}
  \caption{Software-Kosten}
\end{table}

JetBrains-Lizenzen werden benötigt, um dem Team hochwertige Entwicklungsumgebungen zur Verfügung zu stellen. Webhosting, Domain und SSL-Zertifikat sind notwendig, um die entwickelte Anwendung sicher und zugänglich im Internet bereitzustellen.

\begin{table}[ht]
  \centering
  \begin{tabular}{lrr}
    \toprule
    \textbf{Cloud \& Backup} & \textbf{Anzahl} & \textbf{Kosten (EUR)} \\
    \midrule
    Backup-Lösung & 1 & 300 \\
    Cloud-Speicher & 1 & 200 \\
    \midrule
    \textbf{Gesamtkosten Zusatzkosten} & & 500 \\
    \bottomrule
  \end{tabular}
  \caption{Zusatzkosten}
\end{table}

Eine Backup-Lösung ist wichtig, um Datenverlust zu vermeiden und die Sicherheit der Projektdaten zu gewährleisten. Cloud-Speicherung ist erforderlich, um eine zentrale Speicherung und einen einfachen Zugriff auf Projektdateien für alle Teammitglieder zu ermöglichen.

\begin{table}[ht]
  \centering
  \begin{tabular}{lr}
    \toprule
    \textbf{Kostenkategorie} & \textbf{Kosten (EUR)} \\
    \midrule
    Hardware & 6800 \\
    Software & 1760 \\
    Zusatzkosten & 500 \\
    \midrule
    Gesamtkosten Sachmittel & 9060 \\
    \bottomrule
  \end{tabular}
  \caption{Gesamtkosten Sachmittel}
\end{table}

\subsubsection{Gesamtkosten des Projektes}

Die gesamten Projektkosten setzen sich aus den Personalkosten und den Sachmittelkosten zusammen.

\begin{table}[ht]
  \centering
  \begin{tabular}{lr}
    \toprule
    \textbf{Kostenkategorie} & \textbf{Kosten (EUR)} \\
    \midrule
    Personalkosten & 18384 \\
    Sachmittelkosten & 9060 \\
    \midrule
    \textbf{Gesamtkosten des Projektes} & 27444 \\
    \bottomrule
  \end{tabular}
  \caption{Gesamtkosten des Projektes}
\end{table}

\section*{Ergebnis}
Die gesamten Projektkosten betragen \textbf{27.444 EUR}.

\subsection{Wirtschaftlichkeitsberechnung}\label{wirtschaftlichkeitsberechnung}

Bei der Gewinnschwellenberechnung und der Amortisationsrechnung handelt es sich um wichtige Instrumente zur Bewertung der Wirtschaftlichkeit eines Projekts. Diese Berechnungen geben Aufschluss darüber, ab welchem Punkt das Projekt profitabel wird und wie lange es dauert, bis sich die Investitionen amortisiert haben. 

\subsubsection{Gewinnschwellenberechnung}\label{gewinnschwellenberechnung}

Die Gewinnschwellenberechnung bestimmt, ab welcher Menge verkaufter Einheiten das Projekt Gewinn erzielt. Angenommen, das Produktabo wird zu einem Preis von 10 EUR pro Monat angeboten.

\begin{itemize}
    \item \textbf{Gesamtkosten des Projektes:} 27.444 EUR
    \item \textbf{Preis des Produktabos:} 10 EUR pro Monat
    \item \textbf{Gewinnschwelle:} \[ \frac{27.444 \text{ EUR}}{10 \text{ EUR/Monat}} = 2.744,4 \text{ Monate} \]
\end{itemize}

Das bedeutet, es müssen 2.745 Abos für einen Monat verkauft werden, um die Kosten zu decken.

\subsubsection{Amortisationsrechnung}\label{amortisationsrechnung}

Die Amortisationsrechnung berechnet die Zeit, die benötigt wird, um die Investition durch die Einnahmen zu decken. Angenommen, es werden monatlich 500 Abos verkauft.

\begin{itemize}
    \item \textbf{Monatliche Einnahmen:} \[ 500 \text{ Abos} \times 10 \text{ EUR/Monat} = 5.000 \text{ EUR/Monat} \]
    \item \textbf{Amortisationszeit:} \[ \frac{27.444 \text{ EUR}}{5.000 \text{ EUR/Monat}} = 5,49 \text{ Monate} \]
\end{itemize}

Das bedeutet, dass es ungefähr 5,5 Monate dauert, bis sich die Investition in das Projekt amortisiert hat.

