% !TEX root = ../projektdokumentation.tex

\section{Wirtschaftliche Betrachtung}\label{wirtschaftliche-betrachtung}

\subsection{Marktuntersuchung}\label{marktuntersuchung}

\subsubsection{Zielgruppe}\label{zielgruppe}

\begin{itemize}
\item
  Beschreibung der Zielgruppe:

  \begin{itemize}
  
  \item
    Welche Erwartungen hat die Zielgruppe an das Produkt?
  \end{itemize}
\end{itemize}

\subsubsection{Marktvolumen und Marktpotential}\label{marktvolumen-und-marktpotential}

\begin{itemize}
\item
  Analyse des Marktvolumens und -potentials:

  \begin{itemize}
  \item
    Wie groß ist der Markt für das Produkt?
  \item
    Welche Wachstumsmöglichkeiten gibt es?
  \end{itemize}
\end{itemize}

\subsubsection{Konkurrenzanalyse}\label{konkurrenzanalyse}

\begin{itemize}
\item
  Analyse der Wettbewerbssituation:

  \begin{itemize}
  
  \item
    Welche Konkurrenzprodukte gibt es und wie unterscheiden sie sich?
  \end{itemize}
\end{itemize}

\subsection{Marketingmix (4P)}\label{marketingmix-4p}

\subsubsection{Produktpolitik}\label{produktpolitik}

\begin{itemize}
\item
  Beschreibung der Produktpolitik:

  \begin{itemize}
  
  \item
    Welche Produktvarianten werden angeboten?
  \end{itemize}
\end{itemize}

\subsubsection{Preispolitik}\label{preispolitik}

\begin{itemize}
\item
  Festlegung der Preispolitik:

  \begin{itemize}
  
  \item
    Welche Preisstrategie wird verfolgt?
  \end{itemize}
\end{itemize}

\subsubsection{Kommunikationspolitik}\label{kommunikationspolitik}

\begin{itemize}
\item
  Beschreibung der Kommunikationsstrategie:

  \begin{itemize}
  
  \item
    Wie wird das Produkt beworben?
  \end{itemize}
\end{itemize}

\subsubsection{Distributionspolitik}\label{distributionspolitik}

\begin{itemize}
\item
  Beschreibung der Distributionspolitik:

  \begin{itemize}
  
  \item
    Wie gelangt das Produkt zum Endkunden?
  \end{itemize}
\end{itemize}

\subsection{Kostenplanung}\label{kostenplanung}

Im Folgenden werden die potenziellen Kosten, die im Laufe des Projekts anfallen können, im Einzelnen erläutert. Dies betrifft die Berechnung der Personalkosten, die jeweiligen Stundensätze, die Kosten für Sachmittel und die Gesamtkosten für die Umsetzung des Projekts.

\subsubsection{Personalkosten}\label{personalkosten}

\section*{Projektdaten}
\begin{tabular}{ll}
  \toprule
  \textbf{Projektstart:} & 14. Februar 2024 \\
  \textbf{Projektende:} & 29. Mai 2024 \\
  \textbf{Arbeitszeit:} & Jeden Mittwoch von 8:10 bis 13:20 mit 2 Pausen je 20 Minuten \\
  \bottomrule
\end{tabular}

\section*{Berechnung der Arbeitszeit}
Die Arbeitszeit pro Mittwoch beträgt 4 Stunden und 30 Minuten (270 Minuten) abzüglich 40 Minuten Pause, also 3 Stunden und 50 Minuten (230 Minuten oder 3.83 Stunden).

\[
  \text{Arbeitszeit pro Woche} = 3.83 \text{ Stunden}
\]

Der Zeitraum von 14. Februar 2024 bis 29. Mai 2024 umfasst insgesamt 16 Wochen.

\section*{Stundensätze}
\begin{tabular}{lr}
  \toprule
  \textbf{Person} & \textbf{Stundensatz (EUR)} \\
  \midrule
  Fabian Reichwald & 50 \\
  Phillipp Batelka & 45 \\
  Justus Sieweke & 55 \\
  Daniel Quellenberg & 40 \\
  Jan Mahnken & 50 \\
  Christopher Spencer & 60 \\
  \bottomrule
\end{tabular}

\section*{Berechnung der Personalkosten}
\[
  \begin{aligned}
    \text{Personalkosten pro Woche} &= (\text{Fabian Reichwald} + \text{Phillipp Batelka} + \text{Justus Sieweke} + \text{Daniel Quellenberg} + \text{Jan Mahnken} + \text{Christopher Spencer}) \times \text{Arbeitszeit pro Woche} \\
    &= (50 + 45 + 55 + 40 + 50 + 60) \times 3.83 \\
    &= 300 \times 3.83 \\
    &= 1149 \text{ EUR/Woche}
  \end{aligned}
\]

\begin{tabular}{lr}
  \toprule
  \textbf{Gesamtkosten pro Woche} & 1149 EUR \\
  \midrule
  \textbf{Anzahl der Wochen} & 16 \\
  \midrule
  \textbf{Gesamtkosten} & 18384 EUR \\
  \bottomrule
\end{tabular}

\section*{Ergebnis}
Die gesamten Personalkosten für 6 Personen über den Zeitraum von 3 Sprints (16 Wochen) betragen \textbf{18.384 EUR}.


\subsubsection{Sachmittelkosten}\label{sachmittelkosten}

\begin{itemize}
\item
  Aufschlüsselung der Sachmittelkosten:

  \begin{itemize}
  
  \item
    Kosten für benötigte Hardware und Software
  \end{itemize}
\end{itemize}

\subsection{Wirtschaftlichkeitsberechnung}\label{wirtschaftlichkeitsberechnung}

\subsubsection{Gewinnschwellenberechnung}\label{gewinnschwellenberechnung}

\begin{itemize}
\item
  Berechnung der Gewinnschwelle:

  \begin{itemize}
  
  \item
    Ab welcher Menge erzielt das Projekt Gewinn?
  \end{itemize}
\end{itemize}

\subsubsection{Amortisationsrechnung}\label{amortisationsrechnung}

\begin{itemize}
\item
  Berechnung der Amortisationszeit:

  \begin{itemize}
  
  \item
    Wie lange dauert es, bis sich das Projekt amortisiert hat?
  \end{itemize}
\end{itemize}