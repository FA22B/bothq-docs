% !TEX root = ../projektdokumentation.tex

\section{Wirtschaftliche Betrachtung}\label{wirtschaftliche-betrachtung}

\subsection{Marktuntersuchung}\label{marktuntersuchung}

\subsubsection{Zielgruppe}\label{zielgruppe}

\subsubsubsection{Die Hauptzielgruppen für BotHQ sind}

\begin{itemize}
  \item \textbf{Gamer und Gaming-Communities:}
  Discord ist bei Gamern sehr beliebt, und viele Gaming-Communities nutzen Bots zur Verwaltung von Servern, zur Bereitstellung von Spielinformationen, zur Organisation von Events und Turnieren usw.

  \item \textbf{Technikbegeisterte und Entwickler:}
  Personen, die sich für Programmierung und Bot-Entwicklung interessieren, aber möglicherweise nicht über die erforderlichen technischen Kenntnisse verfügen, um einen Discord-Bot von Grund auf zu erstellen und zu konfigurieren.

  \item \textbf{Unternehmen und Organisationen:}
  Unternehmen, die Discord für Teamkommunikation nutzen, könnten Bots für Aufgabenautomatisierung, Kundenservice und andere organisatorische Zwecke einsetzen.

  \item \textbf{Content Creator und Streamer:}
  Streamer auf Plattformen wie Twitch und YouTube nutzen häufig Discord-Server zur Interaktion mit ihrer Community und könnten Bots zur Moderation und Engagement-Steigerung einsetzen.
\end{itemize}

\subsubsubsection{Erwartungen der Zielgruppe an BotHQ}

\begin{enumerate}
  \item \textbf{Einfache Installation und Einrichtung:}
  Nutzer erwarten, dass die Installation und Konfiguration eines Discord-Bots über BotHQ einfach und schnell erfolgt, ohne dass umfangreiche technische Kenntnisse erforderlich sind.

  \item \textbf{Benutzerfreundlichkeit:}
  Eine intuitive Benutzeroberfläche und klare Anweisungen sind entscheidend, damit Nutzer ohne technische Vorkenntnisse den Bot und die Plugins problemlos installieren und konfigurieren können.

  \item \textbf{Vielfältige Plugins und Erweiterungen:}
  Die Möglichkeit, eine breite Palette an Plugins und Erweiterungen für den Bot zu nutzen, um verschiedene Funktionen und Bedürfnisse abzudecken, ist wichtig. Dazu gehören Moderation, Musik-Streaming, Spiele, Ankündigungen und mehr.

  \item \textbf{Zuverlässigkeit und Stabilität:}
  Nutzer erwarten, dass die Bots und Plugins stabil und zuverlässig arbeiten, ohne häufige Ausfälle oder Fehler.

  \item \textbf{Sicherheit:}
  Sicherheit ist ein zentrales Anliegen. Nutzer möchten sicherstellen, dass ihre Daten und Server durch den Einsatz des Bots nicht gefährdet werden. BotHQ sollte daher sicherheitsrelevante Updates und Maßnahmen zur Absicherung der Bots bereitstellen.

  \item \textbf{Kundensupport und Dokumentation:}
  Guter Kundensupport und umfangreiche Dokumentation (FAQs, Tutorials, Foren) sind notwendig, um Nutzern bei Problemen oder Fragen schnell und effektiv zu helfen.

  \item \textbf{Flexibilität und Anpassbarkeit:}
  Die Möglichkeit, Bots und Plugins an die spezifischen Bedürfnisse und Präferenzen der Nutzer anzupassen, ist von großem Vorteil. Dies kann durch benutzerdefinierte Konfigurationen und Optionen erreicht werden.

  \item \textbf{Integration mit anderen Diensten:}
  Nutzer könnten erwarten, dass BotHQ eine Integration mit anderen Diensten und Plattformen bietet, z.B. Twitch, YouTube, Twitter oder anderen sozialen Netzwerken, um die Funktionalität des Bots zu erweitern.
\end{enumerate}

\subsubsubsection{Zusammenfassung}

Durch die Berücksichtigung der oben genannten Erwartungen kann BotHQ ein attraktives und wertvolles Produkt für seine Zielgruppe bieten. Eine benutzerfreundliche, sichere und flexible Lösung, die eine einfache Installation und Konfiguration von Discord-Bots ermöglicht, wird die Bedürfnisse von Gamern, Entwicklern, Unternehmen und Content Creators gleichermaßen erfüllen.
Dieser umfassende Ansatz zur Marktuntersuchung und Zielgruppenanalyse stellt sicher, dass BotHQ den Anforderungen und Erwartungen seiner Nutzer gerecht wird, was letztlich zum Erfolg des Produkts beiträgt.

\subsubsection{Marktvolumen und Marktpotential}\label{marktvolumen-und-marktpotential}

\paragraph{Analyse des Marktvolumens}
Um das Marktvolumen für BotHQ zu bestimmen, betrachten wir den aktuellen Markt für Discord-Bots und die Nutzerbasis von Discord:

\begin{itemize}
  \item \textbf{Nutzerbasis von Discord:} Discord hat weltweit über 250 Millionen registrierte Nutzer und täglich mehr als 100 Millionen aktive Nutzer. Die Plattform ist besonders bei Gamern, Technikbegeisterten und Community-Organisatoren beliebt.

  \item \textbf{Verwendung von Discord-Bots:} Eine signifikante Anzahl von Discord-Nutzern verwendet Bots zur Automatisierung von Aufgaben, Verwaltung von Servern und Verbesserung der Nutzererfahrung. Schätzungen zufolge verwenden etwa 30\% der Discord-Server mindestens einen Bot.

  \item \textbf{Marktgröße:} Wenn wir annehmen, dass 30\% der aktiven Discord-Nutzer auf Bots zurückgreifen, ergibt dies ein potenzielles Marktvolumen von etwa 30 Millionen aktiven Nutzern weltweit, die an BotHQ interessiert sein könnten.
\end{itemize}

\paragraph{Wachstumsmöglichkeiten}
Der Markt für Discord-Bots weist erhebliches Wachstumspotential auf. Hier sind einige der wichtigsten Faktoren:

\begin{enumerate}
  \item \textbf{Wachstum der Discord-Plattform:} Discord wächst weiterhin schnell, sowohl in Bezug auf die Nutzerbasis als auch auf die Anzahl der Server. Mit der zunehmenden Verbreitung von Discord in verschiedenen Communities und Branchen steigt auch die Nachfrage nach Bots, die spezifische Aufgaben automatisieren und Server effizienter verwalten können.

  \item \textbf{Steigende Nachfrage nach Automatisierung:} In vielen Bereichen wird der Bedarf an Automatisierungslösungen immer größer. Discord-Bots können eine Vielzahl von Aufgaben automatisieren, von der Moderation bis hin zur Bereitstellung von Informationen und Interaktionen mit der Community.

  \item \textbf{Erweiterte Funktionen und Integrationen:} Mit fortschreitender Technologie und neuen Entwicklungen in der API-Integration können Discord-Bots immer komplexere und nützlichere Funktionen bieten. Dies eröffnet neue Möglichkeiten für Entwickler und erhöht das Interesse und die Nachfrage nach solchen Lösungen.

  \item \textbf{Marktexpansion:} Während Discord ursprünglich hauptsächlich von Gamern genutzt wurde, hat sich die Plattform inzwischen auf andere Bereiche wie Bildung, Unternehmenskommunikation und soziale Netzwerke ausgeweitet. Dies eröffnet neue Marktsegmente, die für BotHQ erschlossen werden können.

  \item \textbf{Innovative Geschäftsmöglichkeiten:} Durch das Angebot von Premium-Diensten, personalisierten Lösungen und Support-Services kann BotHQ zusätzliche Einnahmequellen erschließen und die Bindung zu den Nutzern erhöhen.
\end{enumerate}

\paragraph{Zusammenfassung}
Die Analyse des Marktvolumens und -potentials zeigt, dass BotHQ auf einen großen und wachsenden Markt trifft. Mit über 30 Millionen potenziellen Nutzern und einer zunehmenden Nachfrage nach Automatisierung und erweiterten Funktionen bietet der Markt erhebliche Wachstumsmöglichkeiten. Durch die Erschließung neuer Marktsegmente und die Entwicklung innovativer Geschäftsmodelle kann BotHQ langfristig erfolgreich sein und seine Marktposition stärken.

\subsubsection{Konkurrenzanalyse}\label{konkurrenzanalyse}

\paragraph{Analyse der Wettbewerbssituation}
Im Bereich der Discord-Bot-Management-Plattformen gibt es mehrere Konkurrenzprodukte, die ähnliche Dienstleistungen anbieten. Hier sind einige der wichtigsten Wettbewerber und deren Unterschiede zu BotHQ:

\begin{itemize}
  \item \textbf{Dyno Bot:}
  Dyno Bot ist einer der beliebtesten Discord-Bots mit umfassenden Verwaltungsfunktionen. Er bietet Moderation, Automatisierung und benutzerdefinierte Commands. Dyno ist sehr benutzerfreundlich, jedoch kann die Konfiguration von Plugins und erweiterter Funktionen für Anfänger komplex sein.

  \item \textbf{Mee6:}
  Mee6 ist ein weiterer weit verbreiteter Bot, bekannt für seine einfache Einrichtung und vielfältigen Funktionen wie Moderation, Levels, Musik und Willkommensnachrichten. Die meisten Funktionen sind kostenlos, jedoch bietet Mee6 Premium-Funktionen gegen eine Gebühr an. Mee6 legt einen starken Fokus auf Gamification durch Leveling-Systeme, was BotHQ ebenfalls berücksichtigen könnte.

  \item \textbf{Carl-bot:}
  Carl-bot bietet umfangreiche Moderations- und Automatisierungswerkzeuge, einschließlich Reaktionsrollen und Logging. Er ist für seine Flexibilität bekannt, kann jedoch für neue Benutzer schwer zu navigieren sein. Carl-bot ist besonders stark in der Verwaltung großer Server.

  \item \textbf{GAwesome Bot:}
  GAwesome Bot ist flexibel und erweiterbar, bietet aber weniger spezialisierte Funktionen im Vergleich zu einigen anderen Bots. Es ist ein guter Allrounder, aber weniger fokussiert auf spezifische Funktionen oder Benutzergruppen.

  \item \textbf{Nightbot:}
  Nightbot ist hauptsächlich für Twitch-Streamer konzipiert, bietet jedoch auch Discord-Integration. Seine Hauptstärken liegen in der Integration mit Streaming-Diensten und nicht unbedingt in der umfassenden Serververwaltung.

  \item \textbf{BotGhost:}
  BotGhost ermöglicht die Erstellung benutzerdefinierter Discord-Bots ohne Programmierkenntnisse. Es bietet eine Drag-and-Drop-Oberfläche für die Bot-Erstellung und viele Anpassungsoptionen. BotGhost ist jedoch weniger bekannt und hat eine kleinere Nutzerbasis im Vergleich zu den etablierten Bots.
\end{itemize}

\paragraph{Unterscheidungsmerkmale von BotHQ}
BotHQ bietet mehrere einzigartige Funktionen und Vorteile, die es von der Konkurrenz abheben:

\begin{itemize}
  \item \textbf{Automatisierte Installation und Konfiguration:} Im Gegensatz zu vielen Konkurrenzprodukten, die manuelle Konfigurationen erfordern, bietet BotHQ eine vollständig automatisierte Installation und Konfiguration von Bots und Plugins, was die Benutzerfreundlichkeit erheblich steigert.

  \item \textbf{Integrierte Plugin-Verwaltung:} BotHQ ermöglicht es Nutzern, Plugins einfach zu aktivieren und zu konfigurieren, ohne den Bot neu starten oder manuell Dateien bearbeiten zu müssen. Dies macht die Anpassung und Verwaltung des Bots besonders einfach und effizient.

  \item \textbf{Benutzerfreundliche Oberfläche:} Eine intuitive und leicht zu navigierende Benutzeroberfläche, die auch für Anfänger geeignet ist, stellt sicher, dass Nutzer ohne technische Vorkenntnisse den vollen Funktionsumfang des Bots nutzen können.

  \item \textbf{Sicherheitsfokussierung:} BotHQ legt großen Wert auf Sicherheit und Datenschutz, bietet regelmäßige Updates und Sicherheits-Patches, um sicherzustellen, dass die Bots und Daten der Nutzer geschützt sind.

  \item \textbf{Kundensupport und Community:} BotHQ bietet umfangreichen Kundensupport und eine aktive Community, die Nutzern bei Problemen und Fragen zur Seite steht. Dies umfasst auch detaillierte Dokumentationen und Tutorials.

  \item \textbf{Flexibilität und Erweiterbarkeit:} BotHQ unterstützt eine breite Palette von Plugins und ermöglicht es Entwicklern, eigene Plugins zu erstellen und zu integrieren, wodurch der Funktionsumfang des Bots kontinuierlich erweitert werden kann.
\end{itemize}

\paragraph{Zusammenfassung}
BotHQ tritt in einen Markt mit starken und etablierten Wettbewerbern ein. Durch die Fokussierung auf Benutzerfreundlichkeit, automatisierte Prozesse, Sicherheit und umfangreichen Support kann BotHQ jedoch einzigartige Vorteile bieten und sich von der Konkurrenz abheben. Dies macht BotHQ zu einer attraktiven Wahl für Nutzer, die eine einfache und sichere Lösung zur Verwaltung ihrer Discord-Bots suchen.


\subsection{Marketingmix (4P)}\label{marketingmix-4p}

\subsubsection{Produktpolitik}\label{produktpolitik}

\begin{itemize}
\item
  Beschreibung der Produktpolitik:

  \begin{itemize}
  
  \item
    Welche Produktvarianten werden angeboten?
  \end{itemize}
\end{itemize}

\subsubsection{Preispolitik}\label{preispolitik}

\begin{itemize}
\item
  Festlegung der Preispolitik:

  \begin{itemize}
  
  \item
    Welche Preisstrategie wird verfolgt?
  \end{itemize}
\end{itemize}

\subsubsection{Kommunikationspolitik}\label{kommunikationspolitik}

\begin{itemize}
\item
  Beschreibung der Kommunikationsstrategie:

  \begin{itemize}
  
  \item
    Wie wird das Produkt beworben?
  \end{itemize}
\end{itemize}

\subsubsection{Distributionspolitik}\label{distributionspolitik}

\begin{itemize}
\item
  Beschreibung der Distributionspolitik:

  \begin{itemize}
  
  \item
    Wie gelangt das Produkt zum Endkunden?
  \end{itemize}
\end{itemize}

\subsection{Kostenplanung}\label{kostenplanung}

Im Folgenden werden die potenziellen Kosten, die im Laufe des Projekts anfallen können, im Einzelnen erläutert. Dies betrifft die Berechnung der Personalkosten, die jeweiligen Stundensätze, die Kosten für Sachmittel und die Gesamtkosten für die Umsetzung des Projekts.

\subsubsection{Personalkosten}\label{personalkosten}

\section*{Projektdaten}
\begin{tabular}{ll}
  \toprule
  \textbf{Projektstart:} & 14. Februar 2024 \\
  \textbf{Projektende:} & 29. Mai 2024 \\
  \textbf{Arbeitszeit:} & Jeden Mittwoch von 8:10 bis 13:20 mit 2 Pausen je 20 Minuten \\
  \bottomrule
\end{tabular}

\section*{Berechnung der Arbeitszeit}
Die Arbeitszeit pro Mittwoch beträgt 4 Stunden und 30 Minuten (270 Minuten) abzüglich 40 Minuten Pause, also 3 Stunden und 50 Minuten (230 Minuten oder 3.83 Stunden).

\[
  \text{Arbeitszeit pro Woche} = 3.83 \text{ Stunden}
\]

Der Zeitraum von 14. Februar 2024 bis 29. Mai 2024 umfasst insgesamt 16 Wochen.

\section*{Stundensätze}
\begin{tabular}{lr}
  \toprule
  \textbf{Person} & \textbf{Stundensatz (EUR)} \\
  \midrule
  Fabian Reichwald & 50 \\
  Phillipp Batelka & 45 \\
  Justus Sieweke & 55 \\
  Daniel Quellenberg & 40 \\
  Jan Mahnken & 50 \\
  Christopher Spencer & 60 \\
  \bottomrule
\end{tabular}

\section*{Berechnung der Personalkosten}
\[
  \begin{aligned}
    \text{Personalkosten pro Woche} &= (\text{Fabian Reichwald} + \text{Phillipp Batelka} + \text{Justus Sieweke} \\
    &\quad + \text{Daniel Quellenberg} + \text{Jan Mahnken} + \text{Christopher Spencer}) \\
    &\quad \times \text{Arbeitszeit pro Woche} \\
    &= (50 + 45 + 55 + 40 + 50 + 60) \times 3.83 \\
    &= 300 \times 3.83 \\
    &= 1149 \text{ EUR/Woche}
  \end{aligned}
\]

\begin{tabular}{lr}
  \toprule
  \textbf{Gesamtkosten pro Woche} & 1149 EUR \\
  \midrule
  \textbf{Anzahl der Wochen} & 16 \\
  \midrule
  \textbf{Gesamtkosten} & 18384 EUR \\
  \bottomrule
\end{tabular}

\section*{Ergebnis}
Die gesamten Personalkosten für 6 Personen über den Zeitraum von 3 Sprints (16 Wochen) betragen \textbf{18.384 EUR}.

\section*{Gesamtkosten des Projektes}

\subsection*{Personalkosten}
\begin{tabular}{lr}
  \toprule
  \textbf{Beschreibung} & \textbf{Kosten (EUR)} \\
  \midrule
  Gesamtkosten für 6 Personen über 16 Wochen & 18384 \\
  \bottomrule
\end{tabular}

\subsection*{Sachmittelkosten}
\begin{tabular}{lr}
  \toprule
  \textbf{Beschreibung} & \textbf{Kosten (EUR)} \\
  \midrule
  Hardware & 6800 \\
  Software & 1760 \\
  Zusatzkosten & 500 \\
  \midrule
  Gesamtkosten Sachmittel & 9060 \\
  \bottomrule
\end{tabular}

\subsection*{Gesamtkosten}
\begin{tabular}{lr}
  \toprule
  \textbf{Kostenkategorie} & \textbf{Kosten (EUR)} \\
  \midrule
  Personalkosten & 18384 \\
  Sachmittelkosten & 9060 \\
  \midrule
  \textbf{Gesamtkosten des Projektes} & 27444 \\
  \bottomrule
\end{tabular}

\section*{Ergebnis}
Die gesamten Projektkosten betragen \textbf{27.444 EUR}.


\subsubsection{Sachmittelkosten}\label{sachmittelkosten}

\section*{Sachmittelkosten}

\subsection*{Hardware-Kosten}
\begin{tabular}{lrr}
  \toprule
  \textbf{Item} & \textbf{Anzahl} & \textbf{Kosten (EUR)} \\
  \midrule
  Laptop (Windows) & 2 & 2000 \\
  Laptop (Linux) & 2 & 1800 \\
  Laptop (Mac) & 2 & 3000 \\
  \midrule
  \textbf{Gesamtkosten Hardware} & & 6800 \\
  \bottomrule
\end{tabular}

\subsubsection*{Beschreibung}
Wir benötigen 6 Laptops, um den Entwicklern flexible und leistungsfähige Arbeitsgeräte zur Verfügung zu stellen. Unterschiedliche Betriebssysteme (Windows, Linux, Mac) sind erforderlich, um sicherzustellen, dass die entwickelten Anwendungen auf verschiedenen Plattformen getestet und optimiert werden können.

\subsection*{Software-Kosten}
\begin{tabular}{lrr}
  \toprule
  \textbf{Item} & \textbf{Anzahl} & \textbf{Kosten (EUR)} \\
  \midrule
  JetBrains Lizenz (Full Package) & 6 & 1500 \\
  Webhosting & 1 & 200 \\
  Domain & 1 & 10 \\
  SSL-Zertifikat & 1 & 50 \\
  \midrule
  \textbf{Gesamtkosten Software} & & 1760 \\
  \bottomrule
\end{tabular}

\subsubsection*{Beschreibung}
JetBrains-Lizenzen sind notwendig, um hochwertige Entwicklungsumgebungen für das Team bereitzustellen. Webhosting, Domain und SSL-Zertifikat sind erforderlich, um die entwickelte Anwendung sicher und zugänglich im Internet bereitzustellen.

\subsection*{Zusätzliche Kosten}
\begin{tabular}{lrr}
  \toprule
  \textbf{Item} & \textbf{Anzahl} & \textbf{Kosten (EUR)} \\
  \midrule
  Backup-Lösung & 1 & 300 \\
  Cloud-Speicher & 1 & 200 \\
  \midrule
  \textbf{Gesamtkosten Zusatzkosten} & & 500 \\
  \bottomrule
\end{tabular}

\subsubsection*{Beschreibung}
Eine Backup-Lösung ist wichtig, um Datenverlust zu verhindern und die Sicherheit der Projektdaten zu gewährleisten. Cloud-Speicher wird benötigt, um eine zentrale Ablage und einfachen Zugriff auf Projektdateien für alle Teammitglieder zu ermöglichen.

\subsection*{Gesamtkosten}
\begin{tabular}{lr}
  \toprule
  \textbf{Kostenkategorie} & \textbf{Kosten (EUR)} \\
  \midrule
  Hardware & 6800 \\
  Software & 1760 \\
  Zusatzkosten & 500 \\
  \midrule
  Gesamtkosten Sachmittel & 9060 \\
  \bottomrule
\end{tabular}

\section*{Gesamtkosten des Projektes}

\subsection*{Personalkosten}
\begin{tabular}{lr}
  \toprule
  \textbf{Beschreibung} & \textbf{Kosten (EUR)} \\
  \midrule
  Gesamtkosten für 6 Personen über 16 Wochen & 18384 \\
  \bottomrule
\end{tabular}

\subsection*{Sachmittelkosten}
\begin{tabular}{lr}
  \toprule
  \textbf{Beschreibung} & \textbf{Kosten (EUR)} \\
  \midrule
  Hardware & 6800 \\
  Software & 1760 \\
  Zusatzkosten & 500 \\
  \midrule
  Gesamtkosten Sachmittel & 9060 \\
  \bottomrule
\end{tabular}

\subsection*{Gesamtkosten}
\begin{tabular}{lr}
  \toprule
  \textbf{Kostenkategorie} & \textbf{Kosten (EUR)} \\
  \midrule
  Personalkosten & 18384 \\
  Sachmittelkosten & 9060 \\
  \midrule
  \textbf{Gesamtkosten des Projektes} & 27444 \\
  \bottomrule
\end{tabular}

\section*{Ergebnis}
Die gesamten Projektkosten betragen \textbf{27.444 EUR}.


\subsection{Wirtschaftlichkeitsberechnung}\label{wirtschaftlichkeitsberechnung}

\subsubsection{Gewinnschwellenberechnung}\label{gewinnschwellenberechnung}

\begin{itemize}
\item
  Berechnung der Gewinnschwelle:

  \begin{itemize}
  
  \item
    Ab welcher Menge erzielt das Projekt Gewinn?
  \end{itemize}
\end{itemize}

\subsubsection{Amortisationsrechnung}\label{amortisationsrechnung}

\begin{itemize}
\item
  Berechnung der Amortisationszeit:

  \begin{itemize}
  
  \item
    Wie lange dauert es, bis sich das Projekt amortisiert hat?
  \end{itemize}
\end{itemize}