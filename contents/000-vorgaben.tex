\section*{Projektergebnisse}

\begin{itemize}
    \item Produkt
    \item Projektdokumentation
    \begin{itemize}
        \item insgesamt 25 Seiten
        \begin{itemize}
            \item 15 Seiten für das LF 12
            \item 10 Seiten für Wahlpflicht
        \end{itemize}
    \end{itemize}
    \item Präsentation
    \begin{itemize}
        \item Sales-Pitch
        \begin{itemize}
            \item ca. 1:30 -- 2:00 Minuten dynamische Produktpräsentation
        \end{itemize}
        \item ca. 30 Minuten fachbezogene Projektpräsentation
        \item Produktdemo
    \end{itemize}
\end{itemize}

\section*{Technische Hinweise zur Dokumentation}

\begin{itemize}
    \item Die Ausarbeitung enthält:
    \begin{itemize}
        \item Deckblatt
        \item Inhaltsverzeichnis
        \item Abbildungsverzeichnis
        \item Abkürzungsverzeichnis
        \item 25 DIN-A-4 Seiten +/- 10 \% Ausarbeitung
        \begin{itemize}
            \item 15 DIN-A-4 Seiten +/- 10 \% LF12
            \item 10 DIN-A-4 Seiten +/- 10 \% Wahlpflicht
        \end{itemize}
        \item Anhangsverzeichnis
        \item Anhang
    \end{itemize}
    \item Glossar
    \item Aussagekräftiger Quellcode
    \item Literaturverzeichnis
    \item Ein erhöhter/niedrigerer Umfang führt zu Punktabzug (dies gilt auch für unvollständig beschriebene Seiten und den übermäßigen Einsatz von Bildern als Platzfüller)
    \item Schriftart: Times New Roman 12pt/Arial 11pt
    \item Zeilenabstand 1,15pt
    \item Fußnoten 9pt
    \item Seitenränder oben/unten: 3,3 cm
    \item Seitenrand links/rechts: 2,5 cm
    \item Blocksatz - Ausrichtung
    \item Nummerierung der Kapitel
    \item Seitenzahlen
\end{itemize}

\section*{Benotung LF 12}

\begin{itemize}
    \item 40 \% Bewertung des Produkts anhand einer Livedemo, des Git-Links und der Dokumentation
    \item 20 \% Bewertung des Projektmanagements (Product-Backlog, Sprint-Planungen, Durchführung der Retros) anhand des zur Verfügung gestellten Jira-Zugangs
    \item 20 \% Bewertung des technischen Teils der Projektdokumentation
    \item 20 \% Bewertung der Präsentation
\end{itemize}

\section*{Weiterführende Hinweise zum Projektmanagement}

\begin{itemize}
    \item Product-Backlog nach folgenden Kriterien:
    \begin{itemize}
        \item Projekt sinnvoll in User-Stories zerlegt
        \item User-Stories sinnvoll beschrieben
        \item User-Stories sinnvoll bewertet (Effort, Importance)
        \item Akzeptanzkriterien vorhanden und gut gewählt
    \end{itemize}
    \item Projektmanagement mit Scrumdesk:
    \begin{itemize}
        \item Sprintziel gut definiert
        \item Stories zugeordnet und in Tasks zerlegt
        \item Tasks inkl. Zeitschätzung
        \item Eintragen der real gearbeiteten Zeiten
        \item Kommentare bei problematischen Tasks
    \end{itemize}
    \item Projektmanagement im Unterricht:
    \begin{itemize}
        \item Daily-Scrum
        \item Sprint-Planning-Meeting
        \item Sprint-Review-Meeting
        \item Dokumentierte Sprint-Retrospektive
    \end{itemize}
\end{itemize}

\section*{Benotung Wahlpflicht}

\begin{itemize}
    \item 50 \% Bewertung des kaufmännischen Teils der Projektdokumentation (Einleitung, Projektbeschreibung, Soll-Ist-Analyse, Marktforschung, Ressourcenplanung, Wirtschaftlichkeitsbetrachtung) sowie den formalen Anforderungen an die Dokumentation
    \item 35 \% Bewertung der Präsentation
    \item 15 \% Bewertung Sales-Pitch
\end{itemize}

\section*{Bewertung der Projektdokumentation}

\begin{longtable}[]{@{}p{0.7\textwidth}p{0.2\textwidth}@{}}
\toprule
\textbf{Kriterium} & \textbf{Gewichtung} \\
\midrule
\endhead
\textbf{1. Formale Gestaltung} & \\
- Titelblatt / Inhaltsverzeichnis, Literaturverzeichnis, Glossar, etc. & 5 \% \\
- technische Ausführung (Seitenränder, Zeilenabstand, Absätze, etc.) & 5 \% \\
- Rechtschreibung / Zeichensetzung & 5 \% \\
- Ausdruck / Fachsprache / Zitierweise & 5 \% \\
\textbf{2. Inhaltliche Gestaltung} & \\
- Logik des Dokumentaufbaus (Ausgangslage, Aufgabenstellung, Projektumfeld, Schnittstellen) und Vollständigkeit & 15 \% \\
- Problemdarstellung und Vorgehensweise / Methodik & 15 \% \\
- fachliche Kompetenz & 30 \% \\
- Lösungsvorschlag / Begründung der Entscheidung und/oder Änderung & 20 \% \\
\end{longtable}

\section*{Präsentationsbewertung}

\begin{enumerate}
    \item Aufbau und Gliederung des Vortrags (20 \%)
    \begin{itemize}
        \item Vollständigkeit und logische Reihenfolge
    \end{itemize}
    \item Gestaltung der Präsentationsunterlagen (10 \%)
    \begin{itemize}
        \item Lesbarkeit
        \item Seitenzahlen
        \item Farbgebung
        \item Übersichtlichkeit
    \end{itemize}
    \item Darstellung der Thematik (30 \%)
    \begin{itemize}
        \item inhaltliche Qualität
    \end{itemize}
    \item Verwendung von Fachbegriffen (10 \%)
    \begin{itemize}
        \item sinnvolle und richtige Verwendung
    \end{itemize}
    \item Sinnvoller Medieneinsatz (10 \%)
    \begin{itemize}
        \item angemessene dynamische Elemente
        \item z. B. zoomen in größere Diagramme oder Screenshots
    \end{itemize}
    \item Sprachstil, Auftreten, Körpersprache (10 \%)
    \begin{itemize}
        \item z. B. Blickkontakt, Hände, natürliche Sprache, nicht ablesen
    \end{itemize}
    \item Einhaltung des zeitlichen Rahmens (10 \%)
    \begin{itemize}
        \item pro 30 Sekunden Abweichung 10\% Abzug bei diesem Punkt
        \item Bei 5 Minuten Überziehung Abbruch der Präsentation
    \end{itemize}
\end{enumerate}