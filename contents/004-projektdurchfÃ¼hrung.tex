% !TEX root = ../projektdokumentation.tex

\section{Projektdurchführung}\label{projektdurchfuxfchrung}

% \section{Entwurfsphase}
% \subsection{Zielplattform}
% \subsection{Architekturdesign}
% \subsection{Entwurf der Benutzeroberfläche}
% \subsection{Datenmodell}
% \subsection{Geschäftslogik}
% \subsection{Maßnahmen zur Qualitätssicherung}
% \subsection{Pflichtenheft/Datenverarbeitungskonzept}

\subsection{Vorgehensmodell}\label{vorgehensmodell}

Für die Projektdurchführung wurde das agile Vorgehensmodell \gls{Scrum} gewählt. Scrum ist ein weit verbreitetes Framework für die agile Softwareentwicklung, das iterative und inkrementelle Prozesse unterstützt. Es besteht aus festen Rollen, Ereignissen und Artefakten, die bei der iterativen Softwareentwicklung unterstützen.

Die Entscheidung für Scrum wurde aus mehreren Gründen getroffen. Erstens ermöglicht Scrum eine flexible und anpassungsfähige Entwicklung, die besonders nützlich ist, wenn sich Anforderungen im Laufe des Projekts ändern können. Zweitens fördert Scrum eine enge Zusammenarbeit und kontinuierliche Kommunikation im Team, was zu einer besseren Koordination und höherer Produktivität führt. Drittens erleichtert die iterative Natur von Scrum eine regelmäßige Überprüfung und Anpassung des Projekts, was zu einer höheren Qualität des Bot-Framework geführt hat.

Der Methodologie von Scrum folgend wurde das Projekt in drei Sprints unterteilt, wobei jeder Sprint eine feste Dauer von vier Wochen hatte. Dies ermöglichte es dem Team, sich auf kurzfristige Ziele zu konzentrieren und regelmäßig Fortschritte zu präsentieren. Die in unserem Fall wöchentlichen Stand-up-Meetings (Daily Scrum) förderten die Transparenz und halfen dabei, Hindernisse schnell zu identifizieren und zu beseitigen. Durch die Sprint Reviews konnte kontinuierlich Feedback vom ganzen Team eingeholt und in die nächste Planungsphase integriert werden. Insgesamt führte der Einsatz von Scrum zu einer besseren Planbarkeit, höheren Flexibilität und einer kontinuierlichen Verbesserung der Projektarbeit.

\subsection{Umsetzung}\label{umsetzung}

\subsubsection{Klassenmodell}\label{klassenmodell}

Das Klassenmodell zeigt die wichtigsten Klassen, ihre
Attribute und Methoden sowie die Beziehungen zwischen den Klassen.

Zu den Hauptklassen gehören:





Die Beziehungen zwischen den Entitäten sind durch Assoziationen,
Aggregationen und Kompositionen gekennzeichnet. Zum Beispiel hat die
Klasse \texttt{Server} eine Aggregation von \texttt{Plugin}, was
bedeutet, dass ein Server mehrere Plugins haben kann, aber ein Plugin
ohne einen Server existieren kann.

\subsubsection{Datenhaltung}\label{datenhaltung}

Für die Datenhaltung wurde eine relationale Post\-greSQL-Datenbank erstellt. Die Datenbank enthält Tabellen für Benutzer, Server, Plugins und andere Entitäten.

\begin{itemize}
  \item
        Wahl der Datenhaltung:
  \item
        Beschreibung der Datenbankstruktur und des Datenbankmodells
\end{itemize}

\subsubsection{Datenbankstruktur und
Datenbankmodell}\label{datenbankstruktur-und-datenbankmodell}

Die Datenbank besteht aus vier Haupttabellen: \texttt{users},
\texttt{plugins}, \texttt{servers} und \texttt{server\_plugins}. Jede
dieser Tabellen speichert spezifische Informationen, die für den Betrieb
und die Verwaltung der Anwendung notwendig sind. Nachfolgend werden die
Tabellen und ihre Spalten detailliert beschrieben, gefolgt von einer
Darstellung der Beziehungen zwischen den Tabellen.

\paragraph{\texorpdfstring{Tabelle
\texttt{users}}{Tabelle users}}\label{tabelle-users}

Die Tabelle \texttt{users} speichert Informationen über die Benutzer der
Anwendung. Jede Zeile repräsentiert einen einzelnen Benutzer.

\begin{itemize}
\item
  \textbf{id} (INTEGER, PRIMARY KEY, AUTOINCREMENT): Eindeutige
  Identifikationsnummer des Benutzers. Dies ist der Primärschlüssel der
  Tabelle und wird automatisch inkrementiert.
\item
  \textbf{username} (TEXT, UNIQUE, NOT NULL): Der Benutzername des
  Nutzers, der zur Anmeldung verwendet wird. Dieser muss eindeutig sein.
\item
  \textbf{password} (TEXT, NOT NULL): Der Hashwert des Passworts des
  Benutzers. Aus Sicherheitsgründen wird das Passwort nicht im Klartext
  gespeichert.
\item
  \textbf{roles} (TEXT): Rollen des Benutzers (z.B. Administrator,
  Benutzer). Diese Spalte kann mehrere Rollen in einem bestimmten Format
  enthalten (z.B. durch Kommata getrennt).
\end{itemize}

\paragraph{\texorpdfstring{Tabelle
\texttt{plugins}}{Tabelle plugins}}\label{tabelle-plugins}

Die Tabelle \texttt{plugins} enthält Informationen über die
verschiedenen Plugins, die in der Anwendung verwendet werden können.
Jedes Plugin stellt eine eigenständige Erweiterung der Funktionalität
dar.

\begin{itemize}
\item
  \textbf{id} (INTEGER, PRIMARY KEY, AUTOINCREMENT): Eindeutige
  Identifikationsnummer des Plugins. Dies ist der Primärschlüssel der
  Tabelle und wird automatisch inkrementiert.
\item
  \textbf{name} (TEXT, UNIQUE, NOT NULL): Der Name des Plugins.
\item
  \textbf{description} (TEXT, NOT NULL): Eine Beschreibung des Plugins,
  die dessen Funktionalität erklärt.
\item
  \textbf{enabled} (BOOLEAN, NOT NULL, DEFAULT FALSE): Ein Flag, das
  angibt, ob das Plugin aktiviert (TRUE) oder deaktiviert (FALSE) ist.
\item
  \textbf{version} (TEXT, NOT NULL): Die Versionsnummer des Plugins.
\item
  \textbf{created\_at} (DATETIME, DEFAULT CURRENT\_TIMESTAMP): Das Datum
  und die Uhrzeit, wann das Plugin erstellt wurde.
\end{itemize}

\paragraph{\texorpdfstring{Tabelle
\texttt{servers}}{Tabelle servers}}\label{tabelle-servers}

Die Tabelle \texttt{servers} speichert Informationen über die
Discord-Server, auf denen der Bot installiert ist.

\begin{itemize}
\item
  \textbf{id} (INTEGER, PRIMARY KEY, AUTOINCREMENT): Eindeutige
  Identifikationsnummer des Servers. Dies ist der Primärschlüssel der
  Tabelle und wird automatisch inkrementiert.
\item
  \textbf{name} (TEXT, NOT NULL): Der Name des Discord-Servers.
\item
  \textbf{owner\_id} (INTEGER, NOT NULL, FOREIGN KEY REFERENCES
  users(id)): Die ID des Benutzers, der der Besitzer des Servers ist.
  Diese Spalte verweist auf \texttt{users.id}.
\item
  \textbf{created\_at} (DATETIME, DEFAULT CURRENT\_TIMESTAMP): Das Datum
  und die Uhrzeit, wann der Server in der Datenbank erstellt wurde.
\item
  \textbf{updated\_at} (DATETIME, DEFAULT CURRENT\_TIMESTAMP): Das Datum
  und die Uhrzeit, wann der Server zuletzt aktualisiert wurde.
\end{itemize}

\paragraph{\texorpdfstring{Tabelle
\texttt{server\_plugins}}{Tabelle server\_plugins}}\label{tabelle-server_plugins}

Die Tabelle \texttt{server\_plugins} verknüpft die Server und die
Plugins miteinander und speichert spezifische Konfigurationen für jedes
Plugin auf einem Server.

\begin{itemize}
\item
  \textbf{server\_id} (INTEGER, NOT NULL, FOREIGN KEY REFERENCES
  servers(id)): Die ID des Servers, auf dem das Plugin installiert ist.
  Diese Spalte verweist auf \texttt{servers.id}.
\item
  \textbf{plugin\_id} (INTEGER, NOT NULL, FOREIGN KEY REFERENCES
  plugins(id)): Die ID des Plugins, das auf dem Server installiert ist.
  Diese Spalte verweist auf \texttt{plugins.id}.
\item
  \textbf{enabled} (BOOLEAN, NOT NULL, DEFAULT FALSE): Ein Flag, das
  angibt, ob das Plugin auf diesem Server aktiviert (TRUE) oder
  deaktiviert (FALSE) ist.
\item
  \textbf{configurations} (TEXT): Ein JSON-String, der die
  Konfigurationseinstellungen des Plugins speichert. Dieser String
  enthält die spezifischen Einstellungen, die der Benutzer für das
  Plugin auf diesem Server vorgenommen hat.
\end{itemize}

\subsubsection{Beziehungen zwischen den
Tabellen}\label{beziehungen-zwischen-den-tabellen}

Die Tabellen sind durch verschiedene Beziehungen miteinander verknüpft,
um die Datenintegrität zu gewährleisten und die Struktur der Anwendung
abzubilden.

\begin{itemize}
\item
  \textbf{Benutzer und Server}: Ein Benutzer (\texttt{users}) kann
  mehrere Server (\texttt{servers}) besitzen. Dies wird durch die
  Fremdschlüsselbeziehung \texttt{servers.owner\_id} zu
  \texttt{users.id} dargestellt. Ein Benutzer kann somit mehrere Server
  verwalten.
\item
  \textbf{Server und Plugins}: Ein Server (\texttt{servers}) kann
  mehrere Plugins (\texttt{plugins}) verwenden. Diese Beziehung wird
  durch die Verknüpfungstabelle \texttt{server\_plugins} realisiert. Die
  Tabelle \texttt{server\_plugins} enthält die Fremdschlüssel
  \texttt{server\_plugins.server\_id} und
  \texttt{server\_plugins.plugin\_id}, die auf \texttt{servers.id} bzw.
  \texttt{plugins.id} verweisen. Diese Tabelle ermöglicht es, die
  Konfiguration jedes Plugins für jeden Server individuell zu speichern.
\end{itemize}

\subsubsection{Detaillierte Beschreibung der
Beziehungen}\label{detaillierte-beschreibung-der-beziehungen}

\paragraph{Benutzer und Server (1:n
Beziehung)}\label{benutzer-und-server-1n-beziehung}

\begin{itemize}
\item
  \textbf{Primärschlüssel}: \texttt{users.id}
\item
  \textbf{Fremdschlüssel}: \texttt{servers.owner\_id}
\item
  \textbf{Beschreibung}: Ein Benutzer kann mehrere Server besitzen.
  Jeder Server hat einen Besitzer, der in der \texttt{owner\_id}-Spalte
  der \texttt{servers}-Tabelle gespeichert ist.
\end{itemize}

\paragraph{Server und Plugins (n:m
Beziehung)}\label{server-und-plugins-nm-beziehung}

\begin{itemize}
\item
  \textbf{Primärschlüssel}: \texttt{servers.id}, \texttt{plugins.id}
\item
  \textbf{Fremdschlüssel}: \texttt{server\_plugins.server\_id},
  \texttt{server\_plugins.plugin\_id}
\item
  \textbf{Beschreibung}: Ein Server kann mehrere Plugins verwenden, und
  ein Plugin kann auf mehreren Servern verwendet werden. Diese
  n:m-Beziehung wird durch die \texttt{server\_plugins}-Tabelle
  vermittelt, die die IDs der Server und Plugins sowie die
  Konfigurationseinstellungen für die Plugins auf den jeweiligen Servern
  speichert.
\end{itemize}

\subsubsection{Zusammenfassung}\label{zusammenfassung}

Die Datenbankstruktur und das Modell bieten eine robuste Grundlage für
die Verwaltung von Benutzern, Discord-Servern und Plugins. Die klar
definierten Beziehungen zwischen den Tabellen gewährleisten eine hohe
Datenintegrität und ermöglichen eine flexible und erweiterbare
Architektur für die Anwendung. Die Tabelle \texttt{users} speichert
Benutzerinformationen, die Tabelle \texttt{servers} speichert
Serverinformationen, die Tabelle \texttt{plugins} speichert
Plugin-Informationen, und die Verknüpfungstabelle
\texttt{server\_plugins} speichert die Zuordnungen und Konfigurationen
der Plugins zu den jeweiligen Servern.

\subsubsection{Design Patterns}\label{design-patterns}

\begin{itemize}
  \item
        Einsatz von Design Patterns:
  \item
        Welche Design Patterns werden verwendet und warum?
\end{itemize}

\subsection{Qualitätssicherung}\label{qualituxe4tssicherung}

\subsubsection{Teststrategien}\label{teststrategien}

\begin{itemize}
  \item
        Beschreibung der Teststrategien:
  \item
        Welche Tests werden durchgeführt (Unit Tests, Integrationstests, Systemtests)?
\end{itemize}

\subsubsection{Testfälle}\label{testfuxe4lle}

\begin{itemize}
  \item
        Konkrete Testfälle:

        \begin{itemize}

          \item
                Beschreibung der durchgeführten Testfälle und deren Ergebnisse
        \end{itemize}
\end{itemize}